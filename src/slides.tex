%        File: presentation.20150809.tex
%     Created: Fre Sep 04 08:00  2015 C
% Last Change: Fre Sep 04 08:00  2015 C
%

\documentclass{beamer}

\usepackage[french]{babel}
\usepackage[T1]{fontenc}
\usepackage[utf8]{inputenc}
\usepackage{hyperref}
\usepackage{eurosym}
\usepackage{caption}
\usepackage{amssymb,amsmath,amsthm}
\usepackage{colortbl}
\usepackage{tikz}

\usepackage[outputdir=_build]{minted}


%Pour les règles d'inférence.
\usepackage{mathpartir}

\newcommand{\lambdaExpr}[2]{\lambda #1 . \, #2}
\newcommand{\localLetBinding}[3]{let \; #1 \; = \; #2 \; in \; #3}

\definecolor{red}{RGB}{243, 3, 3}
\definecolor{green}{RGB}{3, 243, 3}

\usetheme[navigation]{UMONS}

\begin{document}

\title{Vers un langage typé pour la programmation modulaire}
\date{23 juin 2017}
\institute{UMONS}

\maketitle

\begin{frame}
	\tableofcontents
\end{frame}

\section{Introduction - OCaml}

\begin{frame}
	\begin{center}
		\Huge{Introduction - OCaml}
	\end{center}
\end{frame}

\begin{frame}
  \frametitle{OCaml}
  Commençons par parler d'OCaml...
	\begin{center}
    \begin{itemize}
    \item langage fonctionnel : les fonctions sont des citoyens de première classe.
    \item statiquement typé : les types sont assignés à la compilation, non à
      l'exécution. $\neq$ dynamiquement typé (Python).
    \item dispose d'algorithmes d'inférence de type : pas obligé d'assigner
      manuellement un type à un terme (contrairement à C ou Java).
    \item divisé en deux \og langages \fg :
      \begin{itemize}
      \item langage de base : fonctions, enregistrements, tuples, variables, ...
      \item langage de modules : modules, foncteurs.
      \end{itemize}
    \item \textbf{Problème : }les deux langages possèdent des algorithmes d'inférence de type
      différents et sont gérés différement par le compilateur.
    \end{itemize}
	\end{center}
\end{frame}

%\begin{frame}
%  \inputminted{OCaml}{codes/intro_point2D.ml}
%  \begin{center}
%  \end{center}
%\end{frame}
%
%\begin{frame}
%  \inputminted{OCaml}{codes/intro_point2D_polymorphic.ml}
%  \begin{center}
%  \end{center}
%\end{frame}

\begin{frame}
  \frametitle{But du mémoire}
  \begin{itemize}
  \item Etudier un calcul théorique où les deux langages (et en particulier les
    règles de typage et de sous-typage) sont unifiés. Calcul choisi : DOT (2016).
  \item Représenter les modules et les enregistrements de la même manière.
  \item (Apport principal de ce mémoire). Implémenter un langage de surface, proche d'OCaml, un algorithme de
    typage et un algorithme de sous-typage pour écrire des programmes DOT.
    Langage : RML
  \end{itemize}
\end{frame}

\section{Programmation fonctionnelle et typage}

\begin{frame}
	\begin{center}
		\Huge{Programmation fonctionnelle et typage}
	\end{center}
\end{frame}

\subsection*{Programmation fonctionnelle : $\lambda$-calcul non typé}

\begin{frame}
  \frametitle{$\lambda$-calcul non typé}
  Calcul = syntaxe (termes) + sémantique (règles d'évaluation, non discutée).

  Soit $V$ un ensemble infini dénombrable de variables. La syntaxe du
  $\lambda$-calcul est définie comme le plus petit ensemble $\Lambda$ tel que

  \begin{minipage}{0.45\textwidth}
    % TODO Couleur rouge quand l'autre syntaxe est utilisée.
    \begin{enumerate}
    \item $V \subseteq \Lambda$
    \item $\forall u, v \in \Lambda, u \, v \in \Lambda$
    \item $\forall x \in V, \forall u \in \Lambda, \lambdaExpr{x}{u} \in \Lambda$
    \end{enumerate}
  \end{minipage}
  \begin{minipage}{0.05\textwidth}
    $\Leftrightarrow$
  \end{minipage}
  \begin{minipage}{0.45\textwidth}
    % TODO Couleur verte et peut-être affiché à la fin.
    \begin{align*}
      t ::= & \, & \text{terme} \\
        & \; x & \text{var} \\
        & \; t \, t & \text{app} \\
        & \; \lambdaExpr{x}{t} & \text{abs}
    \end{align*}
  \end{minipage}
  \begin{itemize}
  \item $u \, v \rightarrow$ application d'une fonction $u$ à un paramètre $v$.
  \item $\lambdaExpr{x}{u} \rightarrow$ définition d'une fonction (anonyme) qui
    prend un paramètre $x$ et retourne $u$. 
  \end{itemize}
\end{frame}

%\begin{frame}
%  \frametitle{$\lambda$-calcul non typé - Sémantique}
%\end{frame}

\subsection*{Typage : $\lambda$-calcul simplement typé}

\begin{frame}
  \frametitle{Typage}
  \begin{itemize}
    \item Classer les termes en fonction de leur nature $\rightarrow$ notion de type.
    \item Exemple avec le terme $u \, v$: $u$ doit être une fonction qui prend un
      paramètre $v$ d'un certain type $T_{1}$ et retourne un terme d'un type
      $T_{2}$. On note $T_{1} \rightarrow T_{2}$ le type de $u$ (type flèche).
    \item Formellement et de manière générale, on se donne $\tau$, un ensemble de types, les éléments
      étant notés $T$, et on définit une \textbf{relation de typage}
      entre $\Lambda$ et $\tau$. On note $t : T$ pour dire que $t$ est de type
      $T$. On dit aussi que $t$ est bien typé.
  \end{itemize}
\end{frame}

\begin{frame}
  \frametitle{$\lambda$-calcul simplement typé - Syntaxe des types}
  Soit $\mathcal{B}$ un ensemble de types de base, ses éléments étant notés
  $B$.
  \begin{align*}
    T ::= & \, & \text{types} \\
          & \; B & \text{base} \\
          & \; T \rightarrow T & \text{type des fonctions}
  \end{align*}
\end{frame}

\begin{frame}
  \frametitle{Contexte et jugement de typage, différence de syntaxe}
  \begin{itemize}
  \item $\lambdaExpr{x}{t}$ : type de $x$ ? $\Rightarrow$ on annote la variable
    avec son type. $\lambdaExpr{x}{t}$ devient $\lambdaExpr{x : T}{t}$.
  \item un terme $t$ peut contenir des variables $\Rightarrow$ pour typer $t$, besoin de
    connaître le type de celles-ci $\Rightarrow$ \textbf{contexte ou
      environnement de
    typage} $\Gamma$ = suite $(x_{i}, T_{i})$ où $x_{i}$ est une
    variable et $T_{i}$ est le type de $x_{i}$.
  \item $t : T$ devient $\Gamma \vdash t : T$ (appelé \textbf{jugement de typage}).

  \end{itemize}
\end{frame}

\begin{frame}
  \frametitle{$\lambda$-calcul simplement typé - Règles de typage}
  %Pour typer un terme $t$, on utilise des \textbf{règles de typage}. Une règle de typage
  %permet de dériver des jugements de typage à partir d'autres jugements de
  %typage ou d'axiome.
  \begin{mathpar}
    \inferrule
      {(x : T) \in \Gamma}
      {\Gamma \vdash x : T} \quad (\textsc{T-VAR})
    \and
    \inferrule
      {\Gamma, x : T_{1} \vdash t : T_{2}}
      {\Gamma \vdash \lambdaExpr{x : T_{1}}{t} : T_{1} \rightarrow T_{2}} \quad (\textsc{T-ABS})
    \and
    \inferrule
      {\Gamma \vdash u : T_{1} \rightarrow T_{2} \\ \Gamma \vdash v : T_{1}}
      {\Gamma \vdash u \, v : T_{2}} \quad (\textsc{T-APP})
  \end{mathpar}
\end{frame}
\begin{frame}
  \frametitle{$\lambda$-calcul simplement typé - Arbre de dérivation}
  Une succession d'utilisations de règles de typage mène à \textbf{un arbre de dérivation}.
  \begin{mathpar}
  \inferrule* [Right=(\textsc{T-APP})]
  {\inferrule* [Left=(\textsc{T-VAR})]
    {x : T_{1} \rightarrow T_{2} \in \Gamma}
    {\Gamma \vdash x : T_{1} \rightarrow T_{2}}
    \\
  \inferrule* [Right=(\textsc{T-VAR})]
    {y : T_{1} \in \Gamma}
    {\Gamma \vdash y : T_{1}}
  }
  {\inferrule* [Right=(\textsc{T-ABS})]
    {\Gamma, x : T_{1} \rightarrow T_{2}, y : T_{1} \vdash x \, y : T_{2}}
    {\inferrule* [Right=(\textsc{T-ABS})]
      {\Gamma, x : T_{1} \rightarrow T_{2}
        \vdash
        \lambdaExpr{y : T_{1}}{x \, y} :
        T_{1} \rightarrow T_{2}
      } 
      {\Gamma \vdash \lambdaExpr{x : T_{1} \rightarrow T_{2}}{\lambdaExpr{y :
            T_{1}}{x \, y}} : (T_{1} \rightarrow
        T_{2}) \rightarrow T_{1} \rightarrow T_{2}
      }
    }
  } 
  \end{mathpar}
  Pour le $\lambda$-calcul simplement typé : il existe au plus un arbre de
  dérivation. Pas toujours le cas (exemple : DOT).
\end{frame}

\begin{frame}
  \frametitle{$\lambda$-calcul simplement typé - Sûreté}
  Théorèmes importants :
  \begin{enumerate}
  \item Préservation : si $t : T$ et $t \rightarrow t'$, alors $t' : T$.
  \item Progression : si $t : T$, alors soit $t$ est une
    valeur, soit $t \rightarrow t'$.
  \end{enumerate}
  Preuves : lemmes intermédiaires + induction sur la dérivation $t : T$ ou $t
  \rightarrow t'$.
\end{frame}

\subsection*{Sous-typage et enregistrements}

\begin{frame}
  \frametitle{Sous-typage}
  \begin{itemize}
  \item But : affiner notre relation de typage en définissant une relation
    binaire sur les types (dite de \textbf{sous-typage}).
  \item Notation : $S <: T$ ($S$ est sous-type de $T$).
  \item Ajout d'une règle de typage qui crée un lien entre la relation de typage et la
    relation de sous-typage.
    \begin{mathpar}
      \inferrule
      {\Gamma \vdash t : S \\ S <: T}
      {\Gamma \vdash t : T} \quad (\textsc{T-SUB})
    \end{mathpar}
  \item La relation est souvent transitive et réflexive:
    \begin{mathpar}
      \inferrule
      {S <: T \\ T <: U}
      {S <: U}
      \quad (\textsc{S-TRANS})
      \and
      \inferrule
      {T <: T}
      {}
      \quad (\textsc{S-REFL})
    \end{mathpar}
  \end{itemize}
\end{frame}

\begin{frame}
  \frametitle{Enregistrement}
  \begin{itemize}
  \item enregistrement = ensemble de labels liés à des termes. Par exemple
    $\left\{ l_{1} = 5 ; l_{2} = \lambdaExpr{x}42 \right\}$.
  \item Les champs ne sont pas mutuellement dépendants!
    \begin{minipage}{0.45\textwidth}
      \begin{align*}
        t ::= & \, & \text{terme} \\
              & \; {\color{gray}x} & {\color{gray}\text{var}} \\
              & \; {\color{gray}t \, t} & {\color{gray}\text{app}} \\
              & \; {\color{gray}\lambdaExpr{x : T}{t}} & {\color{gray}\text{abs}} \\
              & \; \left\{ l_{i} = t_{i} \right\}^{1 \leq i \leq n} & \text{enreg} \\
              & \; t.l & \text{proj}
      \end{align*}
    \end{minipage}
    \begin{minipage}{0.45\textwidth}
      \begin{align*}
        T ::= & \, & \text{type} \\
              & \; {\color{gray}T \rightarrow T} & {\color{gray}\text{fonction}} \\
              & \; \left\{ l_{i} : T_{i} \right\}^{1 \leq i \leq n} & \text{enreg}
      \end{align*}
    \end{minipage}
  \end{itemize}
\end{frame}

\begin{frame}
  \frametitle{Sous-typage et enregistrement}
  \begin{mathpar}
    \inferrule
    {\left\{ l_{i} : T_{i} \right\}^{1 \leq i \leq n + k} <: \left\{ l_{i} :
        T_{i} \right\}^{1 \leq i \leq n}}
    {}
    \quad (\textsc{S-RCD-WIDTH})
    \and
    \inferrule
    {\left\{ k_{j} : S_{j} \right\}^{1 \leq j \leq n} \text{ permutation de }
      \left\{ l_{i} : T_{i} \right\} ^ {1 \leq i \leq n}}
    { \left\{ k_{j} : S_{j} \right\} ^ {1 \leq j \leq n} <:
      \left\{ l_{i} : T_{i} \right\} ^ {1 \leq i \leq n}}
    \quad (\textsc{S-RCD-PERM})
    \and
    \inferrule
    {\forall i \in \left\{1, ..., n \right\}, S_{i} <: T_{i}}
    {\left\{ l_{i} : S_{i} \right\}^{1 \leq i \leq n} <: \left\{ l_{i} : T_{i}
      \right\}^{1 \leq i \leq n}}
    \quad (\textsc{S-RCD-DEPTH})
  \end{mathpar}

  (T-SUB) et (S-RCD-WIDTH) $\rightarrow (\lambdaExpr{p : \left\{ x : \mathbb{R}
    \right\}}{p.x}) \left\{ x = \pi ; y = 42 \right\}$ bien typé.
\end{frame}

\section{DOT}

\begin{frame}
	\begin{center}
		\Huge{DOT}
	\end{center}
\end{frame}

\begin{frame}
  \frametitle{DOT - Description}
  \begin{itemize}
  \item Ajoute les types à l'intérieur des enregistrements $\Rightarrow$ types
    dépendants ($x.A$).
  \item Un enregistrement peut avoir des champs mutuellement dépendants grâce à
    une variable interne.
  \item Les fonctions deviennent dépendantes : le type de retour peut dépendre
    du paramètre.
  \end{itemize}
\end{frame}

\begin{frame}
  \frametitle{DOT - Syntaxe des termes}
  \begin{minipage}{0.45\textwidth}
    \begin{align*}
      t ::= & \, & \text{terme} \\
            & \; {\color{gray}x, y} & {\color{gray}\text{var}} \\
            & \; {\color{gray}\lambdaExpr{x : T}{t}} & {\color{gray}\text{abs}} \\
            & \; {\color{red}x \, y} & \text{app} \\
            & \; {\color{red}x}.a & \text{champ proj} \\
            & \; \localLetBinding{x}{t}{t} & \text{let} \\
            & \; \nu(x : T^{x})d & \text{rec} \\
    \end{align*}
  \end{minipage}
  \begin{minipage}{0.45\textwidth}
    \begin{align*}
      d ::= & \, & \text{decl} \\
            & \; \left\{ a = t \right\} & \text{champ} \\
            & \; \left\{ A = T \right\} & \text{type} \\
            & \; d \wedge d & \text{aggregation}
    \end{align*}
  \end{minipage}
\end{frame} 

\begin{frame}
  \frametitle{DOT - Syntaxe des types}
  \begin{align*}
    S, T ::= & \, & \text{type} \\
             & \; Top & \text{top} \\
             & \; Bottom & \text{bottom} \\
             & \; \forall(x : S) T^{x} & \text{fonction} \\
             & \; \left\{ A : S .. T \right\} & \text{type decl} \\
             & \; \left\{ a : T \right\} & \text{champ decl} \\
             & \; S \wedge T & \text{inter} \\
             & \; x.A & \text{type proj} \\
             & \; \mu(x : T^{x}) & \text{rec} \\
  \end{align*}
\end{frame} 


\begin{frame}
  \frametitle{DOT - Quelques règles de typage}
  \begin{center}
    \Huge{...}
  \end{center}
  \begin{mathpar}
    \inferrule
    {\Gamma \vdash x : \left\{ A : S .. T \right\}}
    {\Gamma \vdash S <: x.A}
    \quad (\textsc{<: SEL})
    \and
    \inferrule
    {\Gamma \vdash x : \left\{ A : S .. T \right\}}
    {\Gamma \vdash x.A <: T}
    \quad (\textsc{SEL <:})
    \and
    \inferrule
    {\Gamma \vdash x : T^{x}}
    {\Gamma \vdash x : \mu(z : T^{z})}  \quad (\textsc{VAR-PACK})
    \and
    \inferrule
    {\Gamma \vdash x : \mu(z : T^{z})}
    {\Gamma \vdash x : T^{x}}  \quad (\textsc{VAR-UNPACK})
  \end{mathpar}
  \begin{center}
    \Huge{...}
  \end{center}
\end{frame}

%\begin{frame}
  \frametitle{DOT : règles typage (1/3)}
  \begin{mathpar}
    \inferrule
    {\Gamma, x : T, \Gamma' \vdash x : T}
    {} \quad (\textsc{T-VAR})
    \and
    \inferrule
    {\Gamma \vdash t : S \\ \Gamma \vdash S <: T}
    {\Gamma \vdash t : T} \quad (\textsc{T-SUB})
    \and
    \inferrule
    {\Gamma, x : T \vdash t : U \\ x \notin FV(T)}
    {\Gamma \vdash \lambdaExpr{x : T}{t} : \forall(x : T) U^{x}} \quad (\textsc{ALL-I})
    \and
    \inferrule
    {\Gamma \vdash x : \forall(z : S) T^{z} \\ \Gamma \vdash y : S}
    {\Gamma \vdash x \; y : [z := y] T^{z}} \quad (\textsc{ALL-E})
    \and
    \inferrule
    {\Gamma \vdash t : T \\ \Gamma, x : T \vdash u : U \\ x \notin FV(U)}
    {\Gamma \vdash \localLetBinding{x}{t}{u} : U} \quad (\textsc{LET})
  \end{mathpar}
\end{frame} 

\begin{frame}
  \frametitle{DOT : règles de typage (2/3)}
  \begin{mathpar}
    \inferrule
    {\Gamma \vdash x : T \\ \Gamma \vdash x : U}
    {\Gamma \vdash x : T \wedge U} \quad (\textsc{AND-I})
    \and
    \inferrule
    {\Gamma, x : T \vdash d : T}
    {\Gamma \vdash \nu(x : T^{x})d : \mu(x : T^{x})} \quad (\textsc{$\left\{ \; \right\}$-I})
    \and
    \inferrule
    {\Gamma \vdash x : T^{x}}
    {\Gamma \vdash x : \mu(z : T^{z})}  \quad (\textsc{VAR-PACK})
    \and
    \inferrule
    {\Gamma \vdash x : \mu(z : T^{z})}
    {\Gamma \vdash x : T^{x}}  \quad (\textsc{VAR-UNPACK})
    \and
    \inferrule
    {\Gamma \vdash x : \left\{ a : T \right\}}
    {\Gamma \vdash x.a : T} \quad (\textsc{FLD-E})
  \end{mathpar}
\end{frame}

\begin{frame}
  \frametitle{DOT : règles de typage (3/3)}
  \begin{mathpar}
    \inferrule
    {}
    {\Gamma \vdash \left\{ A = T \right\} : \left\{ A : T  .. T \right\}} \quad (\textsc{TYP-I})
    \and
    \inferrule
    {\Gamma \vdash d_{1} : T_{1} \\ \Gamma \vdash d_{2} : T_{2} \\ dom(d_{1})
      \cap dom(d_{2}) = \emptyset}
    {\Gamma \vdash d_{1} \wedge d_{2} : T_{1} \wedge T_{2}} \quad (\textsc{ANDDEF-I})
    \and
    \inferrule
    {\Gamma \vdash t : T}
    {\Gamma \vdash \left\{ a : t \right\} : \left\{ a : T \right\}} \quad (\textsc{FLD-I})
  \end{mathpar}
\end{frame} 

\begin{frame}
  \frametitle{DOT : règles de sous-typage (1/2)}
  \begin{mathpar}
    \inferrule
    {\Gamma \vdash T <: Top}
    {}
    \quad (\textsc{S-TOP})
    \and
    \inferrule
    {\Gamma \vdash Bottom <: T}
    {}
    \quad (\textsc{S-Bottom})
    \and
    \inferrule
    {\Gamma \vdash S <: T \\ \Gamma \vdash T <: U}
    {\Gamma \vdash S <: U}
    \quad (\textsc{S-TRANS})
    \and
    \inferrule
    {\Gamma \vdash T <: T}
    {}
    \quad (\textsc{S-REFL})
    \and
    \inferrule
    {\Gamma \vdash S_{2} <: S_{1} \\ \Gamma, x : S_{2} \vdash T_{1} <: T_{2}}
    {\Gamma \vdash \forall(x : S_{1}) T_{1} <: \forall(x : S_{2}) T_{2}}
    \quad (\textsc{ALL<:ALL})
    \and
    \inferrule
    {\Gamma \vdash x : \left\{ A : S .. T \right\}}
    {\Gamma \vdash S <: x.A}
    \quad (\textsc{<: SEL})
  \end{mathpar}
\end{frame}

\begin{frame}
  \frametitle{DOT : règles de sous-typage (2/2)}
  \begin{mathpar}
    \inferrule
    {\Gamma \vdash x : \left\{ A : S .. T \right\}}
    {\Gamma \vdash x.A <: T}
    \quad (\textsc{SEL <:})
    \and
    \inferrule
    {\Gamma \vdash T \wedge U <: T}
    {}
    \quad (\textsc{AND-1-<:})
    \and
    \inferrule
    {\Gamma \vdash T \wedge U <: U}
    {}
    \quad (\textsc{AND-2-<:})
    \and
    \inferrule
    {\Gamma \vdash S <: T \\ \Gamma \vdash S <: U}
    {\Gamma \vdash S <: T \wedge U}
    \quad (\textsc{<: AND})
    \and
    \inferrule
    {\Gamma \vdash S_{2} <: S_{1} \\ \Gamma \vdash T_{1} <: T_{2}}
    {\Gamma \vdash \left\{ A : S_{1} .. T_{1} \right\} <: \left\{ A : S_{2} ..
        T_{2} \right\}}
    \quad (\textsc{TYP<:TYP})
    \and
    \inferrule
    {\Gamma \vdash T <: U}
    {\Gamma \vdash \left\{ a : T \right\} <: \left\{ a : U \right\}}
    \quad (\textsc{FLD <: FLD})
  \end{mathpar}
\end{frame}



\begin{frame}
  \frametitle{DOT - Remarques}
  \begin{itemize}
  \item Pas de règles de sous-typage pour les types récursifs $\Rightarrow$
    (VAR-PACK, VAR-UNPACK) + règles de sous-typage.
  \item Règles de sous-typage dépendent du typage (e.g. SEL <: et <: SEL). \textbf{Pas usuel}.
  \item Problème des mauvaises bornes. Si $x : \left\{ A : Top .. Bottom
    \right\} \in \Gamma$, alors $\Gamma \vdash S <: T$ pour tous $S, T$ par (SEL <:),
    (<: SEL) et (S-TRANS).
  \end{itemize}
\end{frame}

\section{Implémentation - RML}

\begin{frame}
	\begin{center}
		\Huge{Implémentation de DOT en OCaml - RML}
	\end{center}
\end{frame}

\begin{frame}
  \frametitle{Implémentation - Remarques}
	\begin{center}
    \begin{enumerate}
      \item Aucun article ne décrit l'implémentation d'algorithmes de typage et
        de sous-typage pour DOT.
      \item Besoin de définir un langage de surface (parseur).
      \item RML implémente des sucres syntaxiques pour écrire plus
        facilement des programmes (fonction à plusieurs arguments, applications
        avec plusieurs arguments), gérés lors du parsing.
      \item RML implémente un algorithme d'inférence de type pour les modules
        (terme $\nu(x)d$ en plus de $\nu(x : T)d$).
        Dans DOT, la définition d'un enregistrement récursif est obligatoirement liée à un type.
      \item RML fournit une librairie standard avec des entiers, des flottants,
        des chaines de caractères, des listes, etc.
      \item Possibilité de voir les arbres de dérivation.
      \end{enumerate}
	\end{center}
\end{frame}

\begin{frame}
  \frametitle{RML - module = enregistrement}
  \inputminted{OCaml}{codes/point2d.rml}
  \begin{itemize}
  \item $\nu(x) d \rightarrow$ struct d end
  \item $\mu('self : T) \rightarrow$ enregistrement
  \item $\left\{ a = t \right\} \rightarrow$ let a = t
  \item $\left\{ A = T \right\} \rightarrow$ type a = t
  \item $\lambdaExpr{x : T_{1}}{t} \rightarrow$ fun(x : $T_{1}$) $\rightarrow$ t
  \end{itemize}
\end{frame}

\begin{frame}
  \frametitle{RML - module = enregistrement - Type}
  \inputminted{OCaml}{codes/point2d.rmli}
\end{frame}

\begin{frame}
  \frametitle{RML - foncteur = fonction}
  \inputminted{OCaml}{codes/makepoint2d.rml}
\end{frame}

\begin{frame}
  \frametitle{RML - foncteur = fonction - Type}
  \inputminted{OCaml}{codes/makepoint2d_sig.rml}
\end{frame}


\begin{frame}
  \frametitle{Implémentation - Pas si facile...}
  Passer de la théorie à l'implémentation n'est pas facile !
  \begin{enumerate}
  \item Gestion des variables libres et liées (utilisation d'AlphaLib).
  \item Un même jugement de typage peut être dérivé de plusieurs manières.
  \item Une même question $S <: T$ peut être posée plusieurs fois (ex: liste
    avec tail).
  \item Indécidabilité du sous-typage $\Rightarrow$ il n'existe pas d'algorithme
    qui termine sur chaque entrée et qui soit correct et complet.
  \end{enumerate}
\end{frame}

%\begin{frame}
%  \frametitle{Implémentation : problèmes}
%  Problèmes rencontrés lors de l'implémentation et non décrits dans les
%  différents papiers:
%  \begin{enumerate}
%  \item DOT n'est pas stable par insertion de binding locaux variable-variable
%  \end{enumerate}
%  \inputminted{OCaml}{codes/terms_binding_variable.rml}
%\end{frame}

\begin{frame}
  \frametitle{Implémentation - Problèmes}
  \begin{enumerate}
  \item DOT n'est pas stable par insertion de binding locaux variable-variable.
    \inputminted{OCaml}{codes/terms_binding_variable.rml}
  \end{enumerate}
\end{frame}

\begin{frame}
  \frametitle{Implémentation - Problèmes}
  \begin{enumerate}
  \item DOT n'est pas stable par insertion de binding locaux variable-variable.
  \item (SEL <:) et (<: SEL) nécessite un algorithme intermédiaire
    (best\_bound) dans le cas où une question $x.A <: y.A$ est posée. Ce dernier
    n'est pas évident !
  \end{enumerate}
    \begin{mathpar}
      \inferrule
      {\Gamma \vdash x : \left\{ A : L .. U\right\}}
      {\Gamma \vdash x.A <: U}
      \quad (\textsc{SEL-<:})
    \end{mathpar}
    devient
    \begin{mathpar}
      \inferrule
      {\Gamma \vdash x : T \\ \Gamma \vdash T <: \left\{ A : L .. U\right\} \\ \Gamma \vdash U
        <: U'}
      {\Gamma \vdash x.A <: U'}
      \quad (\textsc{UN-SEL-<:})
    \end{mathpar}
\end{frame}


\section*{Travail futur}

\begin{frame}
	\begin{center}
		\Huge{Travail futur}
	\end{center}
\end{frame}

\begin{frame}
  \frametitle{Travail futur}
	\begin{center}
    \begin{itemize}
    \item Résoudre le problème des questions déjà posées (en cours, plusieurs idées).
    \item Evaluateur.
    \item Interpréteur interactif.
    \end{itemize}
	\end{center}
\end{frame}

\begin{frame}
  \begin{center}
    \Huge{Merci}
  \end{center}
\end{frame}


%
%\section{Domaines non explorés}
%
%\begin{frame}
%	\begin{center}
%		\Huge{Domaines non explorés}
%	\end{center}
%\end{frame}
%
%
%\begin{frame}
%  \begin{itemize}
%  \item Théorie des catégories.
%  \item Théorie des types.
%  \item Théorie homotopique des types.
%  \item Sémantique dénotationnelle.
%  \end{itemize}
%\end{frame}

\end{document}